\documentclass[a4paper,12pt]{scrreprt}
    %% Used for changing geometry of the page
    %% Cover page text cannot overlay cover sketching/style 
    %% https://ctan.org/pkg/geometry?lang=en
\usepackage{geometry}
    %% Changes language of some packages protocols
    %% e.g., when captioning images: Figure 1. -> Figura 1.
    %% https://ctan.org/pkg/babel?lang=en
\usepackage[portuguese]{babel}
    %% Used for special fonts
    %% Cannot be compiled with pdflatex
    %% https://ctan.org/pkg/fontspec?lang=en
\usepackage{fontspec}
    %% Arial FONT
    \setmainfont{Arial}

    %% More colors and color options
    %% https://ctan.org/pkg/xcolor?lang=en
    %% https://ctan.org/pkg/colortbl?lang=en
\usepackage{xcolor,colortbl}
    %% More tabular options, like dashed/dotted lines
    %% https://ctan.org/pkg/arydshln?lang=en
\usepackage{arydshln}
    %% List of acronyms
    %% https://ctan.org/pkg/nomencl?lang=en
\usepackage[intoc]{nomencl}
    %% Must be called to init nomencl environment  
    \makenomenclature
    %% More images options/settings
    %% https://ctan.org/pkg/graphicx?lang=en
\usepackage{graphics}
    %% Defining subdirectories to image path enviornment
    %% \graphicspath{{sub1}{sub2}...{subN}}
    \graphicspath{{images}}
    
    %% used to handle cross-referencing commands in LaTeX to produce hypertext links in the document
    %% https://ctan.org/pkg/hyperref?lang=en
\usepackage{hyperref}
    %% math environments
    %% https://ctan.org/pkg/amsmath?lang=en

    %% settings
    \hypersetup{
        colorlinks,
        citecolor=black,
        filecolor=black,
        linkcolor=black,
        urlcolor=black
    }

\usepackage{amsmath}
    %% Defining backgrouns, used to make the cover
    %% https://ctan.org/pkg/background?lang=en
\usepackage[some]{background}
    %% Used to make drawings or complex graphics
    %% http://pgf.sourceforge.net/pgf_CVS.pdf
\usepackage{tikz}
    %% Tikz library to point operations ((x1,y1) + (x2,y2))
    \usetikzlibrary{calc}

\usepackage{enumitem}
\usepackage{float}

%% Defining sfdefault font and default font for document
\renewcommand{\familydefault}{\sfdefault}


%% Costume made cover 
%% From there you can use \makecover command to build the cover
\include{cover}

%% Command for using tabs
\newcommand{\tab}{
    \hspace{1cm}}

%==========================================================================
% DOCUMENT
%==========================================================================

\begin{document}

\pagenumbering{gobble}

% builds the cover
\makecover

%% smaller footer and header size
\newgeometry{top=3cm,left=3cm,right=3cm,bottom=4cm}

%==========================================================================
% BEGIN OPCIONAL DEDICATÓRIA
%==========================================================================
\iffalse
\clearpage
\begin{center}
    \thispagestyle{empty}
    \vspace*{\fill}

    \vspace*{\fill}
\end{center}
\clearpage
\fi
%==========================================================================
% END OPCIONAL DEDICATÓRIA
%==========================================================================


%==========================================================================
% BEGIN ABSTRACT PAGE
%==========================================================================



%% Abstract name: \Large font size, flushed left and paragraph skip before abstract content
\renewenvironment{abstract}
 {\par\noindent\textbf{\Large\abstractname}\par\bigskip}
 {}

\begin{flushleft}
\begin{abstract}
    \tab Este projeto surgiu com o intuito de auxiliar as pessoas na proteção do seu património, especialmente aquele situado nas zonas rurais, face aos incêndios florestais que tanto devastam a paisagem portuguesa. Desta forma, poderão ser evitados danos maiores e inesperados a terrenos ou habitações situadas em zonas de floresta.
    
    \tab Assim, será criada uma aplicação que alerta os seus utilizadores em caso de perigo iminente para as suas propriedades, permitindo-lhes atuar em tempo real face aos acontecimentos. Estes poderão, ainda, ter uma visão geral da situação, no território continental, em relação aos incêndios.

    \par \textbf{Área de Aplicação}: Minimização dos danos causados por incêndios florestais
    \par \textbf{Palavras-Chave}: Sistema de Monitorização, Incêndios, Engenharia de Software, Aplicações \textit{Web}, \textit{Microsoft}, \textbf{FIRESAFE}, Fundamentação, Especificação, Construção, Diagramas de \textit{Gantt}, \textit{UML}, Frontend, Backend.
\end{abstract}
\end{flushleft}


\pagebreak

%==========================================================================
% END ABSTRACT PAGE 
%==========================================================================

%==========================================================================
% BEGIN INDEXES PAGES
%==========================================================================

%% Changes table of content name
%% Portuguese babel default : "Conteúdo"
%% Personally I prefer "índice"
\renewcommand{\contentsname}{Índice}

\tableofcontents

\pagebreak

\listoffigures

\pagebreak

\listoftables

\pagebreak

%==========================================================================
% END INDEXES PAGES 
%==========================================================================


%==========================================================================
% BEGIN INTRODUCTION
%==========================================================================

%% Starting page numbering here
\pagenumbering{arabic}

\chapter{Introdução}
    \section{Contextualização}
        \tab O fogo e as maneiras de o obter e de o utilizar de maneira produtiva, foram cruciais para a sobrevivência da Humanidade e permitiram que o Homem iniciasse o  seu caminho em direção à civilização. Este pode surgir de modo indireto, através de catástrofes e eventos naturais, como relâmpagos, vulcões, (etc.), os quais foram os primeiros impactos dos homens primitivos com o fogo e dos quais retiraram as suas propriedades: a luz, o calor e sua capacidade de transmitir a chama a outros materiais. O fogo pode, também, ser adquirido através de métodos diretos e artificiais, como a fricção entre dois paus, o choque entre duas pedras e, até mesmo, com fósforos (como estamos habituados nos dias de hoje). Desconhece-se desde que altura se conseguiu obter fogo artificialmente, no entanto, existem vestígios e evidências que mostram que o fogo já era utilizado pelo Homem na Idade da Pedra, mais propriamente, no Paleolítico. Assim, percebemos que o fogo foi, e ainda é, fulcral para o ser humano continuar a existir. Ainda assim, nem sempre é usado da melhor forma e podemos, hoje, dizer que somos vítimas de algumas memórias melancólicas criadas pela tal força e exuberância que o fogo pode ter.
        
        \tab A 17 de Junho de 2017, no concelho de Pedrógão Grande, no distrito de Leiria, em Portugal, viveu-se uma grande tragédia, que percorreu os corações de milhões de Portugueses e as bocas de milhões de cidadãos do mundo. Uma tragédia que levou consigo a vida de 66 pessoas, deixando mais de 200 feridas e destruindo cerca de 500 habitações. O fogo destruiu ainda 53 mil hectares de território, 20 mil dos quais de floresta, durante 7 longos dias \cite{incendio_pedrogao}.
        
        \tab É então nesta altura, que, em Portugal, começa a aparecer uma melhoria das medidas de prevenção contra os fogos e uma melhoria e criação de aplicações de controlo e de alerta para possíveis incêndios, por parte de várias empresas. Consigo, nasce a ideia da \textbf{FIRESAFE}, que é uma aplicação de desenvolvimento de software orientada à monitorização de incêndios e de possíveis riscos dos mesmos, de modo a aumentar a segurança de todos os cidadãos e de intensificar a prevenção de possíveis grandes catástrofes.
\clearpage
    \section{Apresentação do Caso de Estudo}
        \tab A \textbf{FIRESAFE} é uma aplicação orientada à monitorização de incêndios e de possíveis riscos dos mesmos acontecerem. Assim, esta irá permitir às pessoas serem devidamente avisadas e poderem tomar as suas precauções face a possíveis desastres.
        
        \tab A implementação deste sistema será feita para a utilização em clientes universais (\textit{Web browsers}) e, deste modo, os utilizadores apenas irão necessitar aceder ao endereço da aplicação através de qualquer browser, disponível no seu computador, smartphone, etc. Posto isto, o utilizador da aplicação é então munido com 2 opções: efetuar o login (ou o registo caso nunca se tenha registado) ou utilizar a aplicação sem efetuar o login. No entanto, caso não seja efetuado o login, o utilizador deixa de ser dotado de algumas funcionalidades que irão ser abordadas seguidamente.
        
        \tab Em qualquer um dos tipos de uso o utilizador possui um conjunto de funcionalidades comum a ambos. Neste momento, podemos consultar um mapa de Portugal Continental onde podemos ver várias informações. Neste mapa, podemos ver quais os incêndios que se encontram nos vários estados (no seu início, a decorrer, em resolução ou até dados como concluídos) em Portugal. Em cada incêndio podemos ainda ver informações climatéricas da zona/região do mesmo e dados relativos ao número de humanos, número de veículos terrestres e número de veículos aéreos em intervenção. Podemos, ainda, obter informações relativas às horas dos respectivos estados do incêndio, bem como de possíveis alturas de alerta.
        
        \tab Além destas funcionalidades, o utilizador pode fazer o seu registo na aplicação, tal como mencionado anteriormente. Neste registo, o utilizador fornece os seus dados e algumas informações pessoais tais como: o nome, um username, o e-mail, um número de telemóvel (opcional), a sua data de nascimento, uma password e outros possíveis dados individuais. Encontrando-se registado no sistema, o utilizador pode, então, realizar o seu login. Efetuado o login, o utilizador tem a possibilidade de adicionar localizações favoritas. Adicionando estas localizações, o utilizador irá ser notificado caso exista um incêndio nas proximidades ou caso o risco de incêndio nos próximos dias seja elevado.

    \section{Motivação e Objectivos}

        \tab Portugal é um país muito fustigado por incêndios florestais, especialmente durante o verão. Em 2019, registaram-se 10832 fogos, totalizando 42084 ha de área ardida \cite{relatorio_fogos_2019}. Embora tenha havido melhorias nos últimos anos, os números ainda são algo preocupantes.
        
        \tab Sendo assim, achamos importante desenvolver uma aplicação capaz de alertar os seus utilizadores da ocorrência de incêndios nas proximidades de localizações do seu interesse, de forma a evitar catástrofes maiores. Já é possível receber alertas através de SMS, no entanto, apenas são enviados quando sucedem incêndios de nível bastante elevado. Para além disso, não são enviados alertas customizados consoante as preferências do utilizador, nem é enviada nenhuma informação quando acontecem pequenos incêndios. Estes pequenos incêndios podem ser menosprezados por este sistema de alertas, no entanto, eles podem ser fatais para habitações, terrenos rurais ou, até mesmo, pelo facto de aumentarem o risco de acidentes face a possíveis rotas que possamos ter no dia a dia. O facto de não termos encontrado nenhuma aplicação com esta finalidade também influenciou a nossa decisão, aspirando a resolução deste problema.
        
        \tab Pretendemos, então, resolver um problema há muito conhecido, mas ainda com poucas soluções. A sua primazia será a customização pessoal, para que cada um possa usufruir da melhor forma possível da aplicação. Alargamos o envio de alertas para todo o tipo de incêndios (mesmo os mais insignificantes) para a maior segurança de todos, ao contrário do que já encontramos hoje, que, ou possuímos poucos alertas, recebendo apenas os que estão relacionados com os casos mais intensos, ou então recebemos uma quantidade de informação em demasia e a informação que nos é mais relevante é perdida. Esperamos poder aumentar a segurança das pessoas num dos aspetos da sua vida, tendo como objetivo deliciar as pessoas com o melhor serviço disponível. Dada a abundância destes acontecimentos no verão, detemos, também, como meta a finalização da construção deste serviço até à data estipulada, fazendo com que a aplicação esteja já disponível no decorrer do próximo que se avizinha.

    \section{Estrutura do Relatório}
        \tab No presente relatório demonstramos a primeira etapa da construção do serviço e da aplicação \textbf{FIRESAFE}, que passa pelo Domínio da Engenharia de Software, com particular ênfase no desenvolvimento de aplicações. A construção do software em questão é orientado, principalmente, ao desenvolvimento de um Sistema de Monitorização de Eventos.
        
        \tab A primeira fase do projeto é denominada de Fundamentação. Nesta etapa são tidos como principais objetivos fundamentar, projetar e gerir o desenvolvimento de um sistema de software.
        
        \tab Dessa forma, começamos por expor uma explicação, exibindo uma contextualização sobre o tema. Após ser mostrado o contexto em que estaremos a trabalhar, são apresentadas todas as motivações para a construção do projeto em questão, bem como a implementação que pretendemos engenhar. Além disso, são ainda expressos todos os objetivos que pretendemos alcançar, tanto a nível de realização da empresa, como da satisfação do utilizador comum.
        
        \tab Seguidamente, são identificadas as justificações, a viabilidade e ainda a utilidade do sistema, onde é explicado, de forma mais prática, as vantagens da utilização e da construção deste sistema, bem como a praticabilidade e a exequibilidade do mesmo.
        
        \tab Após isso, é estabelecida a Identidade do Projeto, onde são identificadas várias características e informações sobre a aplicação. Nestas informações podem estar referências ao nome da empresa, faixa etária de utilização, uma breve descrição, entre outras.
        
        \tab Posto isto, é necessário perceber quais os recursos que serão necessários para podermos efetuar a construção deste sistema e a realização deste projeto com sucesso. Aqui, é descrita a forma como iremos obter todos os dados necessários.

        \tab Expostos os tipos de dados necessários, chega, então, o momento de explicar a forma como esses dados são utilizados, realizando uma Maqueta do Sistema onde são evidenciadas as funcionalidades da aplicação e a forma como estas podem ser feitas.
        
        \tab Como em todos os bons projetos, é preciso perceber que medidas implementar para que este seja realizado com êxito. Assim, são apresentadas medidas de sucesso.
        
        \tab Por fim, é preciso fazer uma gestão de toda esta construção. Esta é uma etapa crucial no âmbito da Engenharia de Software, já que é neste momento que se pode fazer a melhor gestão e o melhor acompanhamento de um projeto de Desenvolvimento de Software. É, então, nesta fase, apresentado um plano de desenvolvimento, onde são divididas as tarefas pelos vários elementos do grupo, de modo a toda a equipa conseguir produzir o melhor resultado possível, sem causar fadiga desnecessária de certos elementos ou acontecerem acidentes com datas de entrega e material atrasado.
        
        \tab Posteriormente a esta fase, serão ainda realizadas mais duas fases. É realizada uma segunda fase denominada Especificação, onde será feita uma análise e serão especificados, de forma completa, todos os requisitos operacionais e funcionais de um sistema de software. Por fim, será realizada uma terceira fase, denominada de Construção, onde iremos ingressar no processo de desenvolver, validar, documentar e instalar sistemas de software.
        
    \section{Justificação do Sistema}
        \tab Os incêndios são das catástrofes naturais mais graves em Portugal, tanto pela elevada frequência com que acontecem, como pelos seus efeitos de destruição e grandes prejuízos económicos e ambientais que trazem. Todos os anos, em Portugal, são registados milhares de fogos, que trazem consequências, muitas vezes fatais, para as famílias e para a sociedade portuguesa. Durante o verão, principalmente, muitas habitações e terrenos são destruídos à custa dos incêndios florestais. Muitas vezes, os donos dessas propriedades podem até só receber essa informação vários dias depois, quando já nada pode ser feito para evitar os danos. Contudo, e apesar de não ser possível evitar totalmente uma catástrofe natural, no caso de haver um melhor controlo destes desastres a sociedade poderia, muitas vezes, salvar os seus bens, retirar os seus animais de casa quando o fogo estivesse próximo, ou até mesmo salvar um idoso que, naturalmente, teria dificuldades em fazê-lo sozinho. Assim, surgiu a ideia da nossa aplicação, que poderá ser útil na prevenção desses casos, pois avisa, em tempo real, potenciais ameaças.

    \section{Utilidade do Sistema}
        \tab O nosso sistema pretende ser uma ajuda na prevenção de catástrofes relacionadas com os incêndios florestais. Como referido anteriormente, a aplicação será capaz de detetar incêndios nas proximidades de locais à escolha do utilizador. O utilizador poderá, ainda, ser notificado caso seja detetado risco elevado de incêndio numa dessas localizações. Uma vez que os dados são atualizados em tempo real, o utilizador pode reagir imediatamente, podendo evitar grandes catástrofes. Noutro ponto de vista, um utilizador pode, por exemplo, querer passar uns dias num belo acampamento rodeado de densa vegetação. Neste caso, o utilizador pode verificar, atempadamente, antes de sequer se deslocar até ao local, se existem incêndios nessa zona ou nas suas imediações, fazendo, depois,  a escolha de ingressar para a sua viagem ou não. Pode ainda visualizar o tipo de risco de incêndios que a zona pretendida terá nos próximos dias. Tudo isto é disponibilizado de forma gratuita, sem quaisquer custos ou obrigações, facilitando ainda mais o processo de uso da aplicação.

    \section{Viabilidade do Sistema}
        \tab Nos dias de hoje, a tecnologia encontra-se bastante avançada e, por isso, tudo aquilo que queremos está muitas das vezes à distância de um clique. Esta aplicação não será exceção. Tal como referido anteriormente, os incêndios florestais trazem molestos problemas à sociedade e é necessária uma solução para tal. A nossa aplicação tem como intuito travar muitos desses danos, com uma ideia inovadora e ainda não implementada, o que nos faz acreditar que será uma aplicação com bastante procura. Podemos usufruir deste serviço através de qualquer browser, em qualquer tipo de dispositivo. No entanto, para um maior proveito, o utilizador deve criar uma conta, registando-se e inserindo dados pessoais que qualquer cidadão no mundo atual possui: um e-mail e um número de telemóvel (opcional), entre outros dados menos fulcrais. Após isso, poderá identificar quais as zonas/localizações a monitorizar. Assim, aquando de um incêndio, o utilizador poderá receber notificações no ecrã (caso esteja com a aplicação aberta) ou por e-mail. Deste modo, podemos estar em qualquer lugar com acesso à Internet que iremos receber um e-mail ou os devidos alertas, consoante as localizações que indicamos, de modo a podermos reagir da forma mais rápida possível. 

    \section{Estabelecimento da Identidade do Projeto}
        \begin{itemize}
            \item {\large \textbf{Nome:}} \textbf{FIRESAFE}
            \item {\large \textbf{Categoria:}} Sistema de Monitorização
            \item {\large \textbf{Idioma:}} Português
            \item {\large \textbf{Faixa Etária:}} Todas as idades
            \item {\large \textbf{Descrição:}} O sistema de monitorização é facilmente acessível através de um Web Browser, sendo que qualquer utilizador consegue ter acesso à aplicação através de qualquer aplicativo que lhes permita acesso a um browser. Tendo feito o acesso é apresentado ao utilizador um mapa de Portugal Continental, interativo, onde este conseguirá observar todos os incêndios ocorridos nas últimas 24h, quer estejam estes ativos, a ser combatidos ou extintos. O utilizador consegue, ainda, saber se determinada área corre risco de incêndio para o próprio dia de visualização ou para o dia seguinte e ainda selecionar um incêndio para obter informações sobre o mesmo. Se este decidir efetuar um registo (ou dar login caso já se tenha registado), consegue, ainda, escolher uma localização para monitorizar com o intuito de receber um alerta caso esta tenha um incêndio ativo nas proximidades ou seja previsto risco elevado de incêndio nessa zona.
            \item {\large \textbf{Empresa:}} NTH371
            \item {\large \textbf{Criadores:}} Bruno Dias, Luís Sousa e Pedro Barbosa
            \item {\large \textbf{Logotipo:}} 
                \begin{figure}[hbt!]
                    \centering
                    \frame{\includegraphics[width=.7\textwidth]{images/Fase1/logoFIRESAFE.png}}
                    \caption{Logotipo}
                \end{figure}
        \end{itemize}
        
        \tab Tendo em conta esta ficha de projeto, a nossa aplicação estará disponível e poderá ser acedida por qualquer pessoa desde que esta tenha acesso a um browser. A nossa empresa, inicialmente, tentará contactar e implementar a aplicação nos diversos corpos de bombeiros do país, sejam estes municipais ou voluntários, pois percebemos que são o principal foco de combate ao problema que tentamos resolver e que, juntamente com a correta utilização da \textbf{FIRESAFE}, irão lutar contra esse mesmo problema que Portugal enfrenta.

    \section{Identificação dos recursos necessários}
        \tab Sendo que a empresa NTH371 teve a ideia de construir uma aplicação para a minimização dos danos causados por incêndios florestais, é então necessário reunir um conjunto de recursos que serão necessários para a concessão da mesma. Dessa forma, e como primeiro instinto, os seus criadores foram em busca de informação através das pessoas que mais sofrem com tais catástrofes, e com as quais lidam todos os dias, ou seja, os bombeiros.

        \tab Desta maneira chegou-se a alguns aspetos que seriam cruciais contemplarmos na nossa aplicação. Os dados tinham de ser atualizados o mais rapidamente possível, uma vez que o fogo se pode espalhar de forma muito rápida e é necessário reagir sempre da forma mais breve possível. Outro aspeto que seria bastante importante é possuir grande informação acerca do incêndio e do local (se se está a alastrar, se está a ser apagado, se está resolvido, se está apenas a iniciar-se e também o seu risco, de baixo a muito elevado). Além disso, é ainda importante o aspeto de poder receber notificações em qualquer lugar, a qualquer instante, de forma imediata.
        
        \tab Posto isto, é necessário escolher as ferramentas a utilizar de maneira a  obtermos o melhor sucesso na construção do sistema. Para a concessão de relatórios e de futuras apresentações, optamos por escolher produtos provenientes do \textit{Microsoft Office}. Como sustento à criação e gestão de futuras bases de dados utilizadas, iremos operar com o sistema da \textit{Microsoft SQL Server}. Seguindo o que vemos como regra neste projeto, iremos também utilizar a \textit{Microsoft .NET C\#} como plataforma de desenvolvimento. Para editor de texto, a equipa decidiu usufruir do \textit{Microsoft Visual Studio} por acharmos ser o mais compatível com a área e o tipo de software com que iremos trabalhar. Ainda é, contudo, necessária uma ferramenta fulcral na conceção desta aplicação. Esta ferramenta será uma \textit{API} que nos permitirá acessar dados a uma certa fonte de informações e de facto dar asas à parte da Monitorização de Eventos. De momento não vemos um uso necessário para a concessão de certas funcionalidades através de comandos de voz, no entanto, se utilizarmos essa função futuramente, pretendemos utilizar também uma \textit{API} da \textit{Microsoft}, denominada \textit{Microsoft Speech Platform}.

    \section{Maqueta do sistema}
        \tab A implementação deste sistema será feita para a utilização em clientes universais (\textit{Web browsers}) e, deste modo, os utilizadores apenas irão necessitar aceder ao endereço da aplicação através de qualquer browser, disponível no seu computador, smartphone, etc. Posto isto, é apresentado um mapa de Portugal Continental interativo, onde se podem ver os incêndios ocorridos nesse território nas últimas 24 horas, estejam eles ativos, em resolução ou extintos. O utilizador pode ainda verificar o risco de incêndio para o próprio dia ou para o dia seguinte consoante cada localização, mais propriamente, por concelho. Dentro de cada incêndio podemos ainda visualizar informações relativas ao mesmo e ao seu combate, ou seja, temperaturas, número de meios utilizados, onde se inserem meios humanos, terrestres e aéreos, entre outros dados.
   
        \tab Caso o utilizador se autentique na aplicação, através de um registo (ou login, caso se tenha registado anteriormente), este pode definir uma localização a monitorizar, recebendo um aviso caso ocorra um incêndio nas proximidades ou caso esteja previsto risco elevado de incêndio. Pode, ainda, eliminar uma localização previamente selecionada. Cada utilizador pode ainda mudar de número de telemóvel, e-mail e outras informações pessoais do seu perfil. Posto isto, o utilizador pode, também, definir o tipo de notificações que pretende receber, ou seja, pode escolher notificações no ecrã, por e-mail ou até mesmo um conjunto das várias opções mencionadas.
        
        \tab Em baixo, apresentamos uns esboços feitos através do Software \textit{Justinmind}, de como estamos a pensar organizar a nossa aplicação, nomeadamente, a página inicial com o mapa interativo de Portugal Continental e um pequeno menu com as informações relativas a um dado incêndio.
        
        \vspace{0.6cm}
        
        \begin{figure}[hbt!]
            \centering
            \frame{\includegraphics[width=.8\textwidth]{images/Fase1/pagina_inicial.png}}
            \caption{Página inicial}
        \end{figure}
        
        \begin{figure}[hbt!]
            \centering
            \frame{\includegraphics[width=.8\textwidth]{images/Fase1/menu_incendio.png}}
            \caption{Menu relativo a um incêndio}
        \end{figure}

    \section{Definição de um conjunto de medidas de sucesso}
        \tab Nos dias que correm, definir um conjunto de medidas de sucesso de um determinado projeto é uma das etapas consideradas fulcrais no desenvolvimento de software nas empresas. Uma avaliação global de tudo aquilo que possa ter corrido mal durante o processo de desenvolvimento torna possível a minimização de erros em futuros projetos e também aumenta as chances de sucesso.
        
        \tab Desta forma, para que o nosso projeto alcance o sucesso desejado, definimos as seguintes medidas. Relativamente ao planeamento, dividimos todos os processos e etapas do trabalho de forma a garantir o cumprimento de todos os prazos de entrega existentes. No que se refere à organização do grupo, confiamos o projeto a uma pessoa que achamos que reúne todas as qualidades necessárias para chefiar a equipa mantendo a harmonia e um bom ritmo de trabalho de todos os elementos, assim como garantir que todos trabalhamos para um determinado objetivo, sendo este comum a todos. No que diz respeito ao método utilizado para garantirmos o reconhecimento e utilização da nossa aplicação, o nosso grupo acredita piamente que é apostando na publicidade que iremos encontrar o nosso maior sucesso. Desta forma, iremos controlar o nosso orçamento restante e gastá-lo a publicitar a nossa aplicação através dos meios que nos forem disponibilizados. Também nos comprometemos a sensibilizar a população portuguesa para o problema dos incêndios, sobre o qual a nossa aplicação incide e pretende minimizar os problemas a este acarretados, e a tentar alcançar os diferentes corpos de bombeiros do país (segundo a \textit{PORDATA}, rondavam um total de 465 no ano de 2019 \cite{corpos_bombeiros}) de forma a tentar uniformizar a utilização da nossa aplicação pelos mesmo, o que achamos que irá facilitar e ajudar de forma incontestável os seus trabalhos. Tendo conseguido alcançar estas medidas, iremos focar-nos de seguida no feedback dado pelos utilizadores da nossa aplicação, de forma a corrigir possíveis falhas encontradas e a atualizá-la de forma a satisfazer toda a procura dos nossos utilizadores.
        
        \tab De uma forma geral, acreditamos que, cumprindo os prazos de entrega através de uma boa divisão do trabalho, fazendo uma boa gestão do orçamento disponibilizado, sensibilizando a população para um problema grave e muitas vezes subestimado em Portugal, alcançando os bombeiros do país e dando a palavra aos utilizadores, tentando ir de encontro às suas necessidades e pedidos, conseguimos reunir todas as medidas necessárias que levarão a nossa aplicação ao sucesso.

    \section{Plano de desenvolvimento}
        \tab Este projeto será desenvolvido em três etapas distintas. A primeira, a \textbf{fundamentação}, consiste em identificar e caracterizar o geral da aplicação a desenvolver. Deste modo, começamos por contextualizar e explicar o nosso caso de estudo na presente sociedade, neste caso, um sistema de monitorização de incêndios. De seguida, apresentamos a nossa motivação e os nossos objetivos para termos escolhido abordar o tópico atrás referido. Também justificamos porque é que o nosso projeto é viável e qual a sua utilidade, em termos de modelo de negócio. Prosseguimos com a identidade do projeto e com a identificação dos recursos necessários para o seu desenvolvimento. Por fim, apresentamos uma maqueta representativa de como esperamos que o trabalho fique, quando finalizado, e a maneira como este irá funcionar, um conjunto de medidas de sucesso que teremos de alcançar para que sejamos bem sucedidos e um plano de desenvolvimento onde explicamos, de forma geral, todas as etapas do processo e distribuímos o trabalho pelos elementos do grupo de forma a ser possível realizar o projeto que temos em mãos.
        
        \tab Na segunda etapa, a \textbf{especificação}, trataremos de fazer uma análise dos requisitos necessários, de forma a criarmos uma base sólida do nosso projeto. Para isto ser possível, iremos reunir o grupo, de forma a perceber o que será realmente necessário existir para o nosso projeto ser bem sucedido. De seguida, começaremos por fazer os Use Cases respetivos aos diferentes casos da nossa aplicação, os Diagramas de Sequência e outros diagramas que achemos relevantes. Depois, dividiremos o trabalho, sendo que nos iremos revezar entre fazer o Diagrama de Classes e conceber os Modelos Lógico e Concetual. Finalmente, terminamos esta fase com a especificação do software utilizado, a documentação do projeto e uma análise global do trabalho realizado até ao momento de término da segunda fase.
        
        \tab A terceira etapa, a \textbf{construção}, é onde iremos proceder à efetiva implementação da nossa aplicação. Nesta etapa começaremos por explicar a forma como organizamos a arquitetura do nosso projeto. De seguida, iremos descrever os diferentes módulos utilizados no nosso trabalho e conceber um plano de desenvolvimento para uma boa implementação do software e a respetiva distribuição do trabalho pelos diferentes elementos do grupo. Após estes passos, estaremos prontos para o passo mais importante e demorado desta terceira etapa, que é a implementação do software. Para finalizar, iremos fazer uma pequena abordagem às ferramentas utilizadas e uma validação geral do software desenvolvido e iremos rever o relatório final que foi feito, sempre acompanhando o desenvolvimento dos diversos passos destas três etapas.
        
        \tab Concluindo, elaboramos este plano de desenvolvimento tendo a perfeita noção do trabalho que nos espera, sendo que seremos um grupo de apenas três elementos e, tendo isto em mente, o nosso plano de desenvolvimento do projeto teve de ser muito pensado e bem distribuído, de forma a conseguirmos realizar todas as etapas dentro do prazo estipulado pelos docentes desta unidade curricular. De seguida seguem os diferentes passos das várias etapas distribuídas pelos elementos do grupo, acompanhadas pelos respetivos Diagramas de \textit{Gantt} para cada uma das fases.
        
        \begin{figure}[hbt!]
            \centering
            \frame{\includegraphics[width=\textwidth]{images/Fase1/plano_f1.png}}
            \caption{Plano da Fase 1}
        \end{figure}
        
        \begin{figure}[hbt!]
            \centering
            \frame{\includegraphics[width=\textwidth]{images/Fase1/plano_f2.png}}
            \caption{Plano da Fase 2}
        \end{figure}
        
        \begin{figure}[hbt!]
            \centering
            \frame{\includegraphics[width=\textwidth]{images/Fase1/plano_f3.png}}
            \caption{Plano da Fase 3}
        \end{figure}

%==========================================================================
% END INTRODUCTION
%==========================================================================


%----------------------------------------------------------------%
\chapter{Análise de Requisitos}

\section{Levantamento de Requisitos}

\subsection{Registo na Aplicação}

\subsubsection{Requisitos de Utilizador}

\begin{enumerate}
    \item O Utilizador tem que se registar na aplicação para poder usufruir de certas funcionalidades da mesma.
\end{enumerate}

\subsubsection{Requisitos de Sistema }

\begin{enumerate}
    \item O Sistema deve solicitar certos dados de um Utilizador, como nome de utilizador, palavra-passe e email para o registo.
    \item O Sistema não deve permitir o registo de Utilizadores com o mesmo nome de utilizador.
    \item O Sistema deve armazenar todos os dados de um Utilizador numa Base de Dados.
\end{enumerate}

\subsection{Autenticação na Aplicação}

\subsubsection{Requisitos de Utilizador}

\begin{enumerate}
    \item O Utilizador deve conseguir autenticar-se na aplicação.
\end{enumerate}

\subsubsection{Requisitos de Sistema }

\begin{enumerate}
    \item O Sistema deve solicitar ao Utilizador o seu nome de utilizador e a sua palavra-passe, para o autenticar.
    \item O Sistema deve validar os dados fornecidos pelo Utilizador, garantindo a validade dos seus dados e uma correta autenticação.
\end{enumerate}

\subsection{Edição de Perfil do Utilizador}

\subsubsection{Requisitos de Utilizador}

\begin{enumerate}
    \item O Utilizador deve conseguir editar os dados do seu perfil.
\end{enumerate}

\subsubsection{Requisitos de Sistema }

\begin{enumerate}
    \item O Sistema não deve permitir que o Utilizador mude o seu nome de utilizador.
    \item O Sistema deve permitir que o Utilizador mude os outros dados pessoais, como palavra-passe e email.
    \item O Sistema deve armazenar todas as alterações efetuadas pelo Utilizador na Base de Dados
\end{enumerate}

\subsection{Edição das Localizações Favoritas}

\subsubsection{Requisitos de Utilizador}

\begin{enumerate}
    \item O Uilizador deve conseguir editar a lista com as suas localizações favoritas.
\end{enumerate}

\subsubsection{Requisitos de Sistema }

\begin{enumerate}
    \item O Sistema deve permitir que o Utilizador edite a sua lista de localizações favoritas.
    \item O Sistema deve armazenar todas as alterações efetuadas pelo Utilizador na Base de Dados.
\end{enumerate}

\clearpage
\subsection{Consulta das Localizações Favoritas}

\subsubsection{Requisitos de Utilizador}

\begin{enumerate}
    \item O Utilizador deve conseguir consultar a lista com as suas localizações favoritas.
\end{enumerate}

\subsubsection{Requisitos de Sistema }

\begin{enumerate}
    \item O Sistema deve permitir que o Utilizador consulte a sua lista de localizações favoritas.
    \item O Sistema deve permitir que apenas o Utilizador consiga aceder à sua lista de localizações favoritas.
\end{enumerate}

\subsection{Edição do tipo de Notificações recebidas}

\subsubsection{Requisitos de Utilizador}

\begin{enumerate}
    \item O Uilizador deve conseguir editar o tipo de notificações de alerta que pretende receber.
\end{enumerate}

\subsubsection{Requisitos de Sistema }

\begin{enumerate}
    \item O Sistema deve permitir que o Utilizador edite o tipo de notificações de alerta que pretende receber.
    \item O Sistema deve armazenar todas as alterações efetuadas pelo Utilizador na Base de Dados.
\end{enumerate}

\clearpage
\subsection{Consultar um incêndio}

\subsubsection{Requisitos de Utilizador}

\begin{enumerate}
    \item O Uilizador deve conseguir consultar um incêndio que esteja a ocorrer ou tenha ocorrido nas últimas 24 horas e todos os seus dados.
\end{enumerate}

\subsubsection{Requisitos de Sistema }

\begin{enumerate}
    \item O Sistema deve permitir escolher um incêndio em específico.
    \item O Sistema deve permitir que o Utilizador consulte os dados sobre qualquer incêndio a decorrer ou que tenha acontecido nas últimas 24 horas.
    \item O Sistema deve permitir que qualquer pessoa (estando com sessão iniciada ou não) tenha acesso a esta informação.
\end{enumerate}

\subsection{Terminar Sessão na Aplicação}

\subsubsection{Requisitos de Utilizador}

\begin{enumerate}
    \item O Utilizador deve conseguir terminar a sua sessão e sair da aplicação.
\end{enumerate}

\subsubsection{Requisitos de Sistema }

\begin{enumerate}
    \item O Sistema deve permitir que o Utilizador consiga terminar a sua sessão e sair da aplicação.
\end{enumerate}

\clearpage
\subsection{Consultar histórico de Incêndios de uma Localização}

\subsubsection{Requisitos de Utilizador}

\begin{enumerate}
    \item O Utilizador deve conseguir consultar uma localização de modo a poder observar o seu histórico de incêndios.
\end{enumerate}

\subsubsection{Requisitos de Sistema }

\begin{enumerate}
    \item O Sistema deve permitir que o Utilizador consiga ver histórico de incêndios de uma dada localização.
    \item O Sistema deve fornecer as datas dos incêndios, bem como o número dos incêndios nesse dia e as coordenadas onde os focos dos mesmos aconteceram.
\end{enumerate}


%----------------------------------------------------------------%
\chapter{Modelação de Domínio}

\begin{figure}[hbt!]
    \centering
    \includegraphics[width=0.875\textwidth]{images/Fase2/03.ModelacaoDeDominio/modeloDominio.png}
    \caption{Diagrama do Modelo de Domínio}
\end{figure}

\vspace{0.3cm}
\tab De forma a criar um modelo abstrato capaz de representar todos os comportamentos e informações da aplicação \textbf{FIRESAFE} desenvolvemos o Modelo de Domínio acima apresentado.

\tab Um Utilizador, aquando do seu registo, tem duas entidades relevantes: o Perfil e os Favoritos. Por sua vez, o seu perfil é constituído por um \textit{Username}, uma \textit{Password}, um \textit{Email} e um número de telemóvel. Nos seus favoritos guardam-se as suas localizações favoritas, contendo um distrito, concelho e freguesia. Tudo isto é essencial para o correto funcionamento da aplicação.

\tab Um Utilizador pode consultar incêndios num Mapa que, também ele, contém os incêndios existentes na área que engloba. Estes são constituídos por cinco entidades relevantes: a Meteorologia, a Localização, as Coordenadas, os Meios Utilizados para o seu combate e o seu Estado. 

\tab Por fim, todas estas entidades, juntamente com os relacionamentos que criam entre si, são a base do funcionamento da nossa aplicação e necessários para o correto funcionamento da mesma.


%----------------------------------------------------------------%
\chapter{Modelo de Use Case}

\section{Diagrama de Use Cases}

\begin{figure}[H]
    \centering
    \includegraphics[width=.9\textwidth]{images/Fase2/04.ModeloDeUseCase/diagramaUseCases.png}
    \caption{Diagrama de Use Cases}
\end{figure}

\section{Atores}

\begin{itemize}
    \item \textbf{Utilizador Não Autenticado}
    
    Representa o Utilizador que ainda não se encontra autenticado na aplicação e que se encontra restringido em termos das funcionalidades a que pode aceder. Este apenas consegue registar-se, realizar o \textit{login} e consultar incêndios.
    
    \item \textbf{Utilizador Autenticado}
    
    Representa o Utilizador que se encontra autenticado na aplicação e que pode aceder às funcionalidades completas. Este pode editar o seu perfil, terminar sessão, consultar a sua lista de localizações favoritas, editar essa mesmas lista, consultar incêndios,  editar o tipo de notificações que recebe e consultar os incêndios ocorridos numa dada localização.
    
\end{itemize}


%----------------------------------------------------------------%
\clearpage
\chapter{Use Cases}

\section{Consulta de Incêndio}

\textbf{Descrição: }Utilizador pretende verificar um incêndio.

\textbf{Ator: }Utilizador Autenticado / Utilizador Não Autenticado

\textbf{Pré-Condição: }Incêndio está a decorrer ou decorreu nas últimas 24 horas

\textbf{Pós-Condição: }Utilizador verifica dados sobre incêndio

\textbf{Fluxo Normal: }
\begin{enumerate}
    \item Utilizador indica que pretende verificar dados sobre um dado incêndio;
    \item Sistema processa o pedido;
    \item Sistema apresenta todos os dados relativos ao incêndio escolhido pelo Utilizador;
\end{enumerate}

\vspace{1.5cm}

\tab Qualquer Utilizador da aplicação consegue consultar um incêndio desde que este esteja a decorrer ou tenha decorrido nas últimas 24 horas.

\clearpage

\section{Registar Utilizador}

\textbf{Descrição: }Utilizador preenche registo de modo a registar-se na aplicação.

\textbf{Ator: }Utilizador não Autenticado

\textbf{Pré-Condição: }Utilizador não se encontra registado

\textbf{Pós-Condição: }Utilizador encontra-se registado com sucesso

\textbf{Fluxo Normal: }
\begin{enumerate}
    \item Utilizador indica que se quer registar na aplicação;
    \item Sistema solicita dados necessários para registo (Nome, Username, E-mail, Password e Número de telemóvel opcional);
    \item Utilizador preenche dados necessários;
    \item Utilizador submete registo;
    \item Sistema verifica que dados se encontram disponíveis;
    \item Sistema regista Utilizador;
\end{enumerate}

\textbf{Fluxo Excepcional 1: [Utilizador cancela registo] (passo 4)}
\begin{enumerate}[label=4.\arabic*]
    \item Utilizador cancela registo;
    \item Sistema não regista Utilizador;
\end{enumerate}

\textbf{Fluxo Excepcional 1: [Dados não estão disponíveis] (passo 5)}
\begin{enumerate}[label=5.\arabic*]
    \item Sistema verifica que dados não se encontram disponíveis;
    \item Sistema não regista Utilizador;
\end{enumerate}

\vspace{1.5cm}

\tab Tal como referido anteriormente, um Utilizador, para poder usufruir de certas funcionalidades da nossa aplicação, terá de efetuar um registo onde irá fornecer o seu \emph{Nome de Utilizador}, a sua \emph{palavra-passe} e o seu \emph{email}.

\clearpage

\section{Login Utilizador}

\textbf{Descrição: }Utilizador inicia sessão na aplicação.

\textbf{Ator: }Utilizador Não Autenticado

\textbf{Pré-Condição: }Nenhum utilizador está logado na aplicação

\textbf{Pós-Condição: }Utilizador está logado na aplicação

\textbf{Fluxo Normal: }
\begin{enumerate}
    \item Utilizador indica que quer iniciar sessão na aplicação;
    \item Sistema solicita dados necessários para fazer login (Username/E-mail e Password);
    \item Utilizador preenche dados necessários;
    \item Utilizador submete pedido de login;
    \item Sistema verifica que dados se encontram válidos;
    \item Sistema permite inicio de sessão ao Utilizador;
\end{enumerate}

\textbf{Fluxo Excepcional 1: [Utilizador cancela login] (passo 4)}
\begin{enumerate}[label=4.\arabic*]
    \item Utilizador cancela login;
    \item Nenhum Utilizador fica logado;
\end{enumerate}

\textbf{Fluxo Excepcional 1: [Dados não estão válidos] (passo 5)}
\begin{enumerate}[label=5.\arabic*]
    \item Sistema verifica que dados não se encontram válidos;
    \item Sistema não permite inicio de sessão ao Utilizador;
\end{enumerate}

\vspace{1.5cm}

\tab Para o Utilizador iniciar sessão na aplicação, apenas precisa de preencher os dados pedidos, neste caso, \emph{Nome de Utilizador} e \emph{palavra-passe}. Caso algum destes dados seja inválido o Utilizador não consegue iniciar a sessão na nossa aplicação, podendo, na mesma, usufruir de algumas funcionalidades existentes que não precisam de sessão iniciada.

\clearpage

\section{Terminar Sessão}

\textbf{Descrição: }Utilizador pretende terminar sessão da aplicação.

\textbf{Ator: }Utilizador Autenticado

\textbf{Pré-Condição: }Utilizador está logado na aplicação

\textbf{Pós-Condição: }Utilizador deixa de estar logado na aplicação

\textbf{Fluxo Normal: }
\begin{enumerate}
    \item Utilizador indica que quer encerrar sessão;
    \item Sistema processa pedido de Utilizador;
    \item Sistema encerra sessão de Utilizador;
\end{enumerate}

\vspace{1.5cm}

\tab Qualquer Utilizador com sessão iniciada na aplicação pode, a qualquer momento, terminar a mesma, passando, assim, a poder usufruir somente de funcionalidades que não requerem sessão iniciada na aplicação.

\clearpage

\section{Edição de Perfil}

\textbf{Descrição: }Utilizador pretende editar/alterar dados do seu perfil.

\textbf{Ator: }Utilizador Autenticado

\textbf{Pré-Condição: }Utilizador está logado na aplicação

\textbf{Pós-Condição: }Utilizador efetua alterações ao perfil com sucesso

\textbf{Fluxo Normal: }
\begin{enumerate}
    \item Utilizador indica que pretende editar o seu perfil;
    \item Sistema apresenta dados do perfil do Utilizador;
    \item Utilizador edita dados à sua escolha;
    \item Utilizador submete as alterações;
    \item Sistema altera dados sobre Utilizador com sucesso;
\end{enumerate}

\textbf{Fluxo Excepcional 1: [Utilizador cancela edição] (passo 4)}
\begin{enumerate}[label=4.\arabic*]
    \item Utilizador cancela edição de perfil;
    \item Sistema não altera nenhum dado sobre Utilizador;
\end{enumerate}

\vspace{1.5cm}

\tab Um Utilizador pode, também, editar o seu perfil, conseguindo, neste caso, alterar o seu \emph{email} ou a sua \emph{palavra-passe}. No entanto, nenhum Utilizador da aplicação consegue modificar o seu \emph{Nome de Utilizador}.

\clearpage

\section{Consulta de Localizações Favoritas}

\textbf{Descrição: }Utilizador pretende verificar as suas localizações favoritas.

\textbf{Ator: }Utilizador Autenticado

\textbf{Pré-Condição: }Utilizador está logado na aplicação

\textbf{Pós-Condição: }Utilizador verifica as suas localizações favoritas

\textbf{Fluxo Normal: }
\begin{enumerate}
    \item Utilizador indica que pretende verificar as suas localizações favoritas;
    \item Sistema processa o pedido;
    \item Sistema apresenta a lista de localizações favoritas do Utilizador;
\end{enumerate}

\vspace{1.5cm}

\tab Cada Utilizador que tem o \textit{Login} efetuado na nossa aplicação consegue aceder à sua lista de localizações favoritas.

\clearpage

\section{Edição de Localizações Favoritas}

\textbf{Descrição: }Utilizador pretende alterar as suas localizações favoritas.

\textbf{Ator: }Utilizador Autenticado

\textbf{Pré-Condição: }Utilizador está logado na aplicação

\textbf{Pós-Condição: }Utilizador altera localizações preferidas

\textbf{Fluxo Normal: }
\begin{enumerate}
    \item Utilizador indica que pretende alterar as suas localizações favoritas;
    \item Sistema apresenta as localizações favoritas do Utilizador;
    \item Utilizador edita localizações (adicionando novas ou removendo antigas);
    \item Utilizador submete as alterações feitas às localizações;
    \item Sistema altera localizações favoritas do Utilizador com sucesso;
\end{enumerate}

\textbf{Fluxo Excepcional 1: [Utilizador cancela edição das localizações] (passo 4)}
\begin{enumerate}[label=4.\arabic*]
    \item Utilizador cancela edição das suas localizações favoritas;
    \item Sistema não altera localizações preferidas do Utilizador;
\end{enumerate}

\vspace{1.5cm}

\tab Um Utilizador que tenha o \textit{Login} efetuado na aplicação consegue aceder à sua lista de localizações favoritas e adicionar novas localizações ou remover antigas.

\clearpage

\section{Edição de Tipo de Notificações}

\textbf{Descrição: }Utilizador pretende alterar tipo de notificaões recebidas.

\textbf{Ator: }Utilizador Autenticado

\textbf{Pré-Condição: }Utilizador está logado na aplicação

\textbf{Pós-Condição: }Utilizador altera tipo de notificações recebidas

\textbf{Fluxo Normal: }
\begin{enumerate}
    \item Utilizador indica que pretende alterar tipo de notificaões recebidas;
    \item Sistema apresenta todos os tipos de notificações (mesmo as que já possuímos);
    \item Utilizador edita tipo de notificaões (adicionando novas ou removendo as que já possuía);
    \item Utilizador submete as alterações feitas às notificações;
    \item Sistema altera tipo de notificaões recebidas por parte do Utilizador com sucesso;
\end{enumerate}

\textbf{Fluxo Excepcional 1: [Utilizador cancela edição do tipo de notificações recebidas] (passo 4)}
\begin{enumerate}[label=4.\arabic*]
    \item Utilizador cancela edição do tipo de notificações recebidas;
    \item Sistema não altera tipo de notificações recebidas por parte do Utilizador;
\end{enumerate}

\vspace{1.5cm}

\tab Qualquer Utilizador que tenha efetuado o \textit{Login} na aplicação consegue editar a forma como deseja receber as suas notificações.

\clearpage

\section{Consulta de Histórico de Incêndios numa dada Localização}

\textbf{Descrição: }Utilizador pretende verificar o histórico de incêndios numa dada Localização.

\textbf{Ator: }Utilizador Autenticado

\textbf{Pré-Condição: }Utilizador está logado na aplicação

\textbf{Pós-Condição: }Utilizador verifica histórico de incêndios de uma localização

\textbf{Fluxo Normal: }
\begin{enumerate}
    \item Utilizador indica que pretende verificar o histórico de incêndios numa dada Localização;
    \item Sistema processa o pedido;
    \item Sistema pede localização para verificar o seu histórico de incêndios;
    \item Utilizador insere localização que pretende verificar;
    \item Sistema processa o pedido;
    \item Sistema apresenta histórico de incêndios da localização escolhida;
\end{enumerate}

\vspace{1.5cm}

\tab Cada Utilizador que tem o \textit{Login} efetuado na nossa aplicação consegue aceder e visualizar a lista do histórico de incêndios numa dada localização. Caso estes dados comecem a congestionar o sistema, devido ao espaço ocupado na Base de Dados, estes podem ser limpos periodicamente (por exemplo passado um mês da ocorrência do incêndio).

%\textbf{Descrição: }

%\textbf{Ator: }

%\textbf{Pré-Condição: }

%\textbf{Pós-Condição: }

%\textbf{Descrição: }


%----------------------------------------------------------------%
\chapter{Diagramas de Atividade}

\tab De forma a perceber melhor como é que o Utilizador pode interagir com a aplicação e o modo como as funcionalidades da mesma atuam consoante cada estado e cada interação que o Utilizador tenha com a mesma, construímos um conjunto de diagramas de atividade. Os diagramas definidos abrangem apenas 3 dos requisitos do sistema, de modo a exemplificar como o mesmo é feito. Aqui iremos demonstrar, passo a passo, os requisitos em estudo, consoante as escolhas do Utilizador e respostas do Sistema. Assim, iremos exemplificar os seguintes requisitos: consulta das Localizações Favoritas, edição dos dados do perfil do Utilizador e registo de um novo Utilizador (aumentando cada vez mais o grau de complexidade).

\vspace{1.5cm}

\begin{figure}[H]
    \centering
    \includegraphics[width=1\textwidth]{images/Fase2/06.DiagramasDeAtividade/1.atividadeConsultarLocalizacoes.png}
    \caption{Diagrama de Atividade - Consulta de Localizações Favoritas}
\end{figure}

\begin{figure}[H]
    \centering
    \includegraphics[width=.9\textwidth]{images/Fase2/06.DiagramasDeAtividade/2.atividadeEditaPerfil.png}
    \caption{Diagrama de Atividade - Edição de Perfil}
\end{figure}

\begin{figure}[H]
    \centering
    \includegraphics[width=.9\textwidth]{images/Fase2/06.DiagramasDeAtividade/3.atividadeRegistaUtilizador.png}
    \caption{Diagrama de Atividade - Registo de novo Utilizador}
\end{figure}


%----------------------------------------------------------------%
\chapter{Diagramas de Sequência de Subsistemas}

\tab Tendo como base os diagramas de use cases e de atividades, os modelos de domínio e os use cases construídos anteriormente, passamos agora à construção e apresentação dos diagramas de sequência de subsistemas. Deste modo pretendemos que seja possível visualizar mais facilmente todo o comportamento da aplicação, bem como todos os passos que a mesma tem de fazer para o bom sucesso e funcionamento desta, em geral.

\tab Os diagramas apresentados de seguida são diagramas de subsistemas, escolhidos pelo grupo em vez dos diagramas de sistemas, pois achamos que a perceção do funcionamento da aplicação é melhor desta forma.

\section{Consulta de Incêndio}

\begin{figure}[hbt!]
    \centering
    \frame{\includegraphics[width=1\textwidth]{images/Fase2/07.DiagramasDeSequenciaDeSubsistemas/1.consultaIncendio.png}}
    \caption{Diagrama de Sequência - Consulta de Incêndio}
\end{figure}

\clearpage

\section{Registar Utilizador}

\begin{figure}[hbt!]
    \centering
    \frame{\includegraphics[width=1\textwidth]{images/Fase2/07.DiagramasDeSequenciaDeSubsistemas/2.registarUtilizador.png}}
    \caption{Diagrama de Sequência - Registar Utilizador}
\end{figure}


\section{Login Utilizador}
\begin{figure}[hbt!]
    \centering
    \frame{\includegraphics[width=1\textwidth]{images/Fase2/07.DiagramasDeSequenciaDeSubsistemas/3.loginUtilizador.png}}
    \caption{Diagrama de Sequência - Login Utilizador}
\end{figure}

\clearpage

\section{Terminar Sessão}

\begin{figure}[hbt!]
    \centering
    \frame{\includegraphics[width=1\textwidth]{images/Fase2/07.DiagramasDeSequenciaDeSubsistemas/4.terminarSessao.png}}
    \caption{Diagrama de Sequência - Terminar Sessão}
\end{figure}

\section{Edição Perfil}

\begin{figure}[hbt!]
    \centering
    \frame{\includegraphics[width=1\textwidth]{images/Fase2/07.DiagramasDeSequenciaDeSubsistemas/5.editarPerfil.png}}
    \caption{Diagrama de Sequência - Edição Perfil}
\end{figure}

\clearpage

\section{Edição de Localizações Favoritas}

\begin{figure}[hbt!]
    \centering
    \frame{\includegraphics[width=1\textwidth]{images/Fase2/07.DiagramasDeSequenciaDeSubsistemas/6.editarLocalizacoes.png}}
    \caption{Diagrama de Sequência - Edição Localizações Favoritas}
\end{figure}

\section{Edição de Tipo de Notificações}

\begin{figure}[hbt!]
    \centering
    \frame{\includegraphics[width=1\textwidth]{images/Fase2/07.DiagramasDeSequenciaDeSubsistemas/7.editarNotificacoes.png}}
    \caption{Diagrama de Sequência - Edição Tipo de Notificações}
\end{figure}

\clearpage

\section{Consulta de Localizações Favoritas}

\begin{figure}[hbt!]
    \centering
    \frame{\includegraphics[width=1\textwidth]{images/Fase2/07.DiagramasDeSequenciaDeSubsistemas/8.consultaLocalizacoes.png}}
    \caption{Diagrama de Sequência - Consulta Localizações Favoritas}
\end{figure}

\section{Consulta de Histórico de Incêndios numa dada Localização}

\begin{figure}[hbt!]
    \centering
    \frame{\includegraphics[width=1\textwidth]{images/Fase2/07.DiagramasDeSequenciaDeSubsistemas/9.consultaHistorico.png}}
    \caption{Diagrama de Sequência - Consulta Histórico de Incêndios numa Localização}
\end{figure}


%----------------------------------------------------------------%
\chapter{Diagrama de Classes}

\begin{figure}[hbt!]
    \centering
    \includegraphics[width=1\textwidth]{images/Fase2/08.DiagramaDeClasses/diagramaClasses.png}
    \caption{Diagrama de Classes}
\end{figure}

\tab A partir da modelação do domínio e da especificação dos use cases, ficam percetíveis as entidades que são passíveis de se tornarem classes no desenvolvimento da nossa aplicação.

\tab Através da observação da imagem acima apresentada, é percetível a existência de 6 classes, sendo elas: Mapa, Incendio, Localizacao, Meteorologia, Utilizador e \textit{FIRESAFE}.

\tab A classe principal do nosso projeto é a \textit{FIRESAFE} e esta contém o número total de Utilizadores e de Incêndios da nossa aplicação, contendo dois \textit{maps} que armazenam os mesmas. Esta é responsável por gerir a nossa aplicação.

\tab A classe Utilizador tem como atributos um \textit{Username}, uma \textit{Password}, um \textit{Email} e um número de telemóvel. Contém também um \textit{map} com as localizações favoritas dos Utilizadores.

\tab A classe Incendio tem como atributos uma localização, os meios de combate ao incêndio representados como \textit{INT's}, um estado, latitude e longitude e a meteorologia.

\tab A classe Mapa tem um \textit{map} que armazena os incêndios existentes.


%----------------------------------------------------------------%
\chapter{Máquinas de Estado}
\tab Passando agora à parte de perceber como a aplicação funciona, mais propriamente, àquilo que é visível para o Utilizador e as funcionalidades que o mesmo pode realizar em cada estado e em cada ponto da aplicação. Para isso, o grupo decidiu construir um diagrama de Máquinas de Estado, onde explicamos detalhadamente como cada Utilizador pode usufrir das funcionalidades da aplicação e a interação que pode ter com a mesma.

\vspace{1.5cm}

\begin{figure}[hbt!]
    \centering
    \includegraphics[width=1\textwidth]{images/Fase2/09.MaquinasDeEstado/diagramaEstado.png}
    \caption{Diagrama de Máquinas de Estado}
\end{figure}


%----------------------------------------------------------------%
\chapter{Base de Dados}

\section{Modelo Concetual de Dados}

\begin{figure}[H]
    \centering
    \frame{\includegraphics[width=\textwidth]{images/Fase2/10.BaseDeDados/modelo_concetual.png}}
    \caption{Diagrama do Modelo Concetual}
\end{figure}

\subsection{Identificar entidades}

\begin{table}[H]
\resizebox{\textwidth}{!}{
\begin{tabular}{|l|p{.3\textwidth}|p{.6\textwidth}|}
\hline
Entidade     & Descrição                                                          & Ocorrência                                                                                                                                                                     \\
\hline
Utilizador   & Termo geral que descreve os utilizadores registados no sistema     & Cada utilizador pode consultar as suas localizações guardadas, editar o perfil, consultar os dados de um incêndio ou consultar os incêndios ocorridos numa dada freguesia      \\
\hline
Incêndio     & Termo geral que define todos os incêndios guardados no sistema     & Cada incêndio está associado a uma Meteorologia, bem como a uma Localização                                                                                                    \\
\hline
Meteorologia & Termo geral que descreve o estado meteorológico de um incêndio     & Cada meteorologia está associada a um Incêndio                                                                                                                                 \\
\hline
Localização  & Termo geral que define todas as freguesias de Portugal Continental & Cada localização pode estar associada a vários utilizadores, caso estes a tenham guardado como localização favorita, ou a vários incêndios, caso tenham ocorrido no seu território \\
\hline
\end{tabular}
}
\caption{Identificação de entidades}
\end{table}

\subsection{Identificar atributos}

\begin{table}[H]
\resizebox{\textwidth}{!}{
\begin{tabular}{|l|l|l|l|}
\hline
Entidade     & Atributos                                                                                                                                & Descrição                                                                                                                                                                                                                                                                              & Opcional                                                                              \\
\hline
Utilizador   & \begin{tabular}[c]{@{}l@{}}Id\\ Email\\ Password\\ Nome\\ Telemóvel\end{tabular}                                                         & \begin{tabular}[c]{@{}l@{}}Identificador do Utilizador\\ Email do Utilizador\\ Password do Utilizador\\ Nome do Utilizador\\ Telemóvel do Utilizador\end{tabular}                                                                                                                      & \begin{tabular}[c]{@{}l@{}}Não\\ Não\\ Não\\ Não\\ Sim\end{tabular}                   \\
\hline
Incêndio     & \begin{tabular}[c]{@{}l@{}}Id\\ Estado\\ Coordenadas\\ Meios\_aereos\\ Meios\_terrestres\\ Meios\_humanos\end{tabular}                   & \begin{tabular}[c]{@{}l@{}}Identificador do Incêndio\\ Estado atual do Incêndio\\ Coordenadas do foco do Incêndio\\ Meios aéreos atualmente no combate ao Incêndio\\ Meios terrestres atualmente no combate ao Incêndio\\ Meios humanos atualmente no combate ao Incêndio\end{tabular} & \begin{tabular}[c]{@{}l@{}}Não\\ Não\\ Não\\ Não\\ Não\\ Não\end{tabular}             \\
\hline
Meteorologia & \begin{tabular}[c]{@{}l@{}}Temp\_atual\\ Temp\_min\\ Temp\_max\\ Vento\_vel\\ Vento\_dir\\ Humidade\\ Pressao\_atm\\ Estado\end{tabular} & \begin{tabular}[c]{@{}l@{}}Temperatura atual\\ Temperatura mínima medida\\ Temperatura máxima medida\\ Velocidade atual do vento\\ Direção atual do vento\\ Humidade atual no ar\\ Pressão atmosférica atual\\ Estado do tempo atual\end{tabular}                                      & \begin{tabular}[c]{@{}l@{}}Não\\ Não\\ Não\\ Não\\ Não\\ Não\\ Não\\ Não\end{tabular} \\
\hline
Localização  & \begin{tabular}[c]{@{}l@{}}Id\\ Distrito\\ Concelho\\ Freguesia\end{tabular}                                                             & \begin{tabular}[c]{@{}l@{}}Identificador da Localização\\ Distrito da Localização\\ Concelho da Localização\\ Freguesia da Localização\end{tabular}                                                                                                                                    & \begin{tabular}[c]{@{}l@{}}Não\\ Não\\ Não\\ Não\end{tabular} \\
\hline
\end{tabular}
}
\caption{Identificação de atributos}
\end{table}

\subsection{Identificar relacionamentos}

\begin{table}[H]
\resizebox{\textwidth}{!}{
\begin{tabular}{|l|l|l|l|l|}
\hline
Entidade   & Multiplicidade & Relacionamento & Multiplicidade & Entidade     \\
\hline
Utilizador & 0..N           & guarda         & 0..N           & Localização  \\
\hline
Incêndio   & 0..N           & acontece em    & 1..1           & Localização  \\
\hline
Incêndio   & 1..1           & tem            & 1..1           & Meteorologia \\
\hline
\end{tabular}
}
\caption{Identificação de relacionamentos}
\end{table}

\subsection{Identificar chaves primárias}

A escolha de chaves primárias seguiu um processo idêntico para cada uma das entidades apresentadas, pois em todas elas optamos por usar um identificador (Id) para a identificar.

\textbf{Chaves primárias:}

\begin{itemize}
    \item Utilizador: Id
    \item Localização: Id
    \item Incêndio: Id
    \item Meteorologia: Id
\end{itemize}

\section{Modelo Lógico de Dados}

\begin{figure}[H]
    \centering
    \frame{\includegraphics[width=\textwidth]{images/Fase2/10.BaseDeDados/modelo_logico.png}}
    \caption{Diagrama do Modelo Lógico}
\end{figure}

\subsection{Construção e validação do modelo de dados lógico}

Para a construção deste modelo ser iniciada, necessitamos de estabelecer os elementos que vamos criar. Temos de perceber quais as entidades a criar, os relacionamentos e os atributos, todos eles presentes no modelo conceptual anteriormente construído. Esta construção pode ser feita faseadamente. A derivação das relações para o modelo lógico irá apenas percorrer passos que sejam necessários e que existam. Assim, teremos os passos apresentados de seguida:

\subsubsection{Entidades fortes}

Uma entidade forte entende-se como uma entidade que possui uma chave primária e que não depende de uma outra entidade para que a mesma exista. Assim, no modelo lógico teremos as seguintes relações, representativas das entidades do modelo conceptual:

\textbf{Utilizador} (Id, Nome, Password, Email, Telemóvel) \\
\textbf{Chave Primária:} Id

\textbf{Localização} (Id, Distrito, Concelho, Freguesia) \\
\textbf{Chave Primária:} Id

\textbf{Meteorologia} (Id, Temp\_atual, Temp\_min, Temp\_max, Vento\_vel, Vento\_dir, Humidade, Pressao\_atm, Estado) \\
\textbf{Chave Primária:} Id

\subsubsection{Entidades fracas}

Uma Entidade Fraca é uma entidade que dependerá de outras existentes, uma vez que individualmente a existência destas não faz qualquer sentido. Esta entidade origina a criação de uma nova relação com todos os seus atributos simples, no entanto, a sua chave primária é a composição da sua chave primária com a chave da entidade forte da qual depende.

\textbf{Incêndio} (Id, Meteorologia, Localizacao, Meios\_humanos, Meios\_terrestres, Meios\_aereos, Latitude, Longitude, Estado) \\
\textbf{Chave Primária:} Id, Meteorologia, Localizacao \\
\textbf{Chave Estrangeira:} Meteorologia \textit{referente a} Meteorologia(Id); Localizacao \textit{referente a} Localizacao(Id)

\subsubsection{Relacionamentos de um para um (1:1)}

A representação de um relacionamento de uma para um (1:1) é feita através da cópia da chave primária da entidade forte para a entidade fraca. Esta chave torna-se, então, uma chave estrangeira da entidade fraca.

\textbf{Incêndio} (Id, Meteorologia, Localizacao, Meios\_humanos, Meios\_terrestres, Meios\_aereos, Latitude, Longitude, Estado) \\
\textbf{Chave Estrangeira:} Meteorologia \textit{referente a} Meteorologia(Id)

\subsubsection{Relacionamentos de um para muitos (1:N)}

A representação de um relacionamento de uma para muitos (1:N) é feita através da cópia da chave primária da entidade de cardinalidade 1, para a entidade de cardinalidade N. Esta chave torna-se, então, uma chave estrangeira da entidade de cardinalidade N.

\textbf{Incêndio} (Id, Meteorologia, Localizacao, Meios\_humanos, Meios\_terrestres, Meios\_aereos, Latitude, Longitude, Estado) \\
\textbf{Chave Estrangeira:} Localizacao \textit{referente a} Localizacao(Id)

\subsubsection{Relacionamentos de muitos para muitos (N:M)}

A representação de um relacionamento de muitos para muitos (N:M) é feita através da criação de um novo relacionamento que contenha as chaves primárias de cada entidade como chaves estrangeiras.

\textbf{Favorito} (Utilizador\_Id, Localizacao\_Id) \\
\textbf{Chave Primária:} Utilizador\_Id, Localizacao\_Id \\
\textbf{Chave Estrangeira:} Utilizador\_Id \textit{referente a} Utilizador(Id); Localizacao\_Id \textit{referente a} Localizacao(Id)

\subsection{Determinar o domínio dos atributos}

Nesta secção vamos abordar os atributos das tabelas presentes na nossa base de dados, indicando o respetivo tipo dos valores.

\textbf{Utilizador}

\begin{itemize}
    \item Id - INT
    \item Nome - VARCHAR(45)
    \item Password - VARCHAR(45)
    \item Email - VARCHAR(45)
    \item Telemóvel - VARCHAR(20)
\end{itemize}

\textbf{Localização}

\begin{itemize}
    \item Id - INT
    \item Distrito - VARCHAR(45)
    \item Concelho - VARCHAR(45)
    \item Freguesia - VARCHAR(45)
\end{itemize}

\textbf{Incêndio}

\begin{itemize}
    \item Id - INT
    \item Meios\_humanos - INT
    \item Meios\_terrestres - INT
    \item Meios\_aereos - INT
    \item Latitude - VARCHAR(45)
    \item Longitude - VARCHAR(45)
    \item Estado - INT
\end{itemize}

\textbf{Meteorologia}

\begin{itemize}
    \item Id - INT
    \item Temp\_atual - FLOAT
    \item Temp\_min - FLOAT
    \item Temp\_max - FLOAT
    \item Vento\_vel - FLOAT
    \item Vento\_dir - FLOAT
    \item Humidade - INT
    \item Pressao\_atm - INT
    \item Estado - VARCHAR(45)
\end{itemize}


%----------------------------------------------------------------%
\chapter{Interface}

\tab Explicada a maneira como toda a aplicação funciona, bem como a base de dados que irá ser futuramente utilizada, passamos agora a apresentar a interface que pretendemos que a aplicação possua. Como foi explicado anteriomente (principalmente na primeira etapa deste projeto), a intenção é que a aplicação seja simples e o mais minimalista possível. O foco passa por poder responder ao Utilizador da forma mais rápida e breve possível, onde a informação que o mesmo pretende seja facilmente percetível e de fácil interpretação e procura. 

\tab Assim, a aplicação é construída e estruturada de modo a ser o mais amigável para o Utilizador possível. Como se trata de uma aplicação para verificar dados de situações de risco, os incêndios, queremos que a mesma seja bastante intuitiva e que mostre os dados necessários e mais importantes sem haver muita vagueação e divagação do conteúdo apresentado. Assim, a página inicial/principal, que é a mais importante e onde consta a informação mais pertinente, é bastante simples e intuitiva. A navegação é feita através do uso de botões e os menus de perfil e de edição de dados do Utilizador são também eles bastante claros e de fácil compreensão. Além disso, pretendemos que a aplicação seja capaz de operar em qualquer telemóvel e, por isso, estar adaptada ao mesmo.

\tab Posto isto, é importante realçar que, por se tratar de uma aplicação já com uma grande dimensão, a interface da mesma está preparada para a inclusão de novas funcionalidades ou de alterações na forma como o sistema opera.

\tab Seguidamente apresentamos \textit{Mockups} das várias páginas da aplicação. 

\clearpage

\section{Página Inicial}

\tab Neste Mockup, podemos ver a página inicial, onde temos o mapa por onde podemos navegar e consultar dados sobre incêndios (como iremos ver seguidamente) e onde podemos também clicar em iniciar sessão.

\vspace{1cm}

\begin{figure}[hbt!]
    \centering
    \frame{\includegraphics[width=1\textwidth]{images/Fase2/11.Interface/1.paginaInicial.png}}
    \caption{Mockup - Página Inicial}
\end{figure}
\clearpage

\section{Login}

\tab Clicando em "Iniciar Sessão", podemos fazer o nosso login. No caso de não termos uma conta, podemos também registar um novo Utilizador.

\vspace{1cm}

\begin{figure}[hbt!]
    \centering
    \frame{\includegraphics[width=1\textwidth]{images/Fase2/11.Interface/2.login.png}}
    \caption{Mockup - Login}
\end{figure}
\clearpage

\section{Registar Utilizador}

\tab Clicando em "Registe-se agora", podemos efetuar um novo Registo de Utilizador. No caso de sucesso, seremos retornados de novo para a página de login. No fim de efetuado um login com sucesso, retornamos para a página inicial.

\vspace{1cm}

\begin{figure}[hbt!]
    \centering
    \frame{\includegraphics[width=1\textwidth]{images/Fase2/11.Interface/3.registar.png}}
    \caption{Mockup - Registar Novo Utilizador}
\end{figure}
\clearpage

\section{Consulta Incêndio + Autenticado}

\tab Retornando à página inicial, podemos também consultar incêndios. No seguinte \textit{mockup} iremos ver a página inicial com um Utilizador já autenticado. Os incêndios, contudo, podem ser consultados sem ser necessário qualquer autenticação. Aqui podemos, ainda, clicar no \textit{username} no canto superior direito de modo a aceder ao nosso perfil.

\vspace{1cm}

\begin{figure}[hbt!]
    \centering
    \frame{\includegraphics[width=1\textwidth]{images/Fase2/11.Interface/4.paginaInicialLogin.png}}
    \caption{Mockup - Página Inicial com Utilizador Autenticado}
\end{figure}
\clearpage

\section{Perfil}

\tab Clicando no nosso \textit{username}, podemos aceder ao nosso perfil, onde iremos encontrar os nossos dados, e ainda onde podemos navegar para outras páginas, de modo a usufruir do serviço de customização da nossa aplicação.

\vspace{1cm}

\begin{figure}[hbt!]
    \centering
    \frame{\includegraphics[width=1\textwidth]{images/Fase2/11.Interface/5.perfil.png}}
    \caption{Mockup - Perfil Utilizador}
\end{figure}
\clearpage

\section{Editar Perfil}

\tab Clicando em "Editar Perfil", podemos aceder à página onde podemos customizar os dados, que a aplicação permite, do nosso Utilizador.

\vspace{1cm}

\begin{figure}[hbt!]
    \centering
    \frame{\includegraphics[width=1\textwidth]{images/Fase2/11.Interface/6.editarPerfil.png}}
    \caption{Mockup - Editar Perfil}
\end{figure}
\clearpage

\section{Editar Localizações Favoritas}

\tab Clicando em "Localizações Favoritas", podemos aceder à página onde podemos customizar as nossas localizações favoritas, onde podemos remover localizações já colocadas como favoritas ou adiconar novas. 

\vspace{1cm}

\begin{figure}[hbt!]
    \centering
    \frame{\includegraphics[width=1\textwidth]{images/Fase2/11.Interface/7.editarLocalizacoes.png}}
    \caption{Mockup - Editar Localizações Favoritas}
\end{figure}
\clearpage

\section{Editar Notificações}

\tab Clicando em "Notificações", podemos aceder à página onde podemos customizar o tipo de notificações que pretendemos receber, onde podemos remover e/ou adiconar novas. 

\vspace{1cm}

\begin{figure}[hbt!]
    \centering
    \frame{\includegraphics[width=1\textwidth]{images/Fase2/11.Interface/8.editarNotificacoes.png}}
    \caption{Mockup - Editar Notificações}
\end{figure}
\clearpage

\section{Consultar Histórico Localização}

\tab Clicando em "Histórico Incêndios", podemos aceder à página onde podemos podemos escolher uma localização para consultar o seu histórico de incêndios.

\vspace{1cm}

\begin{figure}[hbt!]
    \centering
    \frame{\includegraphics[width=1\textwidth]{images/Fase2/11.Interface/9.consultarHistorico.png}}
    \caption{Mockup - Editar Notificações}
\end{figure}
\clearpage

\section{Histórico Incêndios}

\tab Escolhendo uma localização, podemos consultar o histórico de incêndios da mesma, onde consta o número de incêndios em cada dia e as coordenadas dos focos dos mesmos. Esta informação pode, mais tarde, ser limpa periodicamente de forma a não ocupar extensivamente o espaço da Base de Dados.

\vspace{1cm}

\begin{figure}[hbt!]
    \centering
    \frame{\includegraphics[width=1\textwidth]{images/Fase2/11.Interface/10.historicoIncendios.png}}
    \caption{Mockup - Editar Notificações}
\end{figure}
\clearpage

%==========================================================================
% BEGIN CONCLUSÕES E TRABALHO FUTURO
%==========================================================================

\chapter{Implementação}

\tab Neste capítulo vamos abordar a terceira fase do projeto prático onde procedemos à implementação do requisitos e das funcionalides acima descritos. Nesta fase, foi essencial fazer uma pesquisa sobre as ferramentas e os recursos que iríamos utilizar para uma correta utilização dos mesmos.

\section{Ferramentas e Arquiteturas Utilizadas}

\tab A nossa aplicação \textit{web} foi desenvolvida recorrendo à arquitetura \textit{ASP.NET MVC}. Desta forma, dividimos a nossa aplicação em três componentes diferentes: \textit{models}, \textit{views} e \textit{controllers}.

\tab Os \textit{models} são responsáveis por armazenar e recuperar todos os objetos referentes ao estado da nossa \textit{api} na base de dados, ou seja, é responsável por toda a lógica da camada de dados.

\tab As \textit{views} são responsáveis pela interface da aplicação com o utilizador da mesma.

\tab Os \textit{controllers} gerem todas as interações do utilizador com a aplicação. Qualquer \textit{input} dado por um determinado utilizador é processado pelo controlador com a ajuda dos diversos \textit{models} e através das \textit{views} mostra o respetivo \textit{output}.

\tab A utilização desta arquitetura foi muito benéfica pois para além de ser bastante simples permitiu uma boa organização e distribuição do trabalho por todos os elementos do grupo.

\tab As ferramentas escolhidas para a realização do projeto foram o \textbf{\textit{Microsoft SQL Server Management Studio 18}} que nos permitiu gerir a nossa base de dados e o \textbf{\textit{Microsoft Visual Studio 2019}} que funcionou com um \textit{IDE}.

\clearpage

\section{Conexão à Base de Dados}

\tab Através da utilização do \textbf{\textit{Microsoft Visual Studio}} a conexão à base de dados fica facilitada e torna-se bastante simples. Este \textit{IDE} recorre à utilização de um gestor de servidores que permite criar uma conexão à base de dados através do utilizador criado no SMSS e a partir daí escrevemos o código para gerir as diferentes tabelas da base de dados. Para tal, criamos uma classe DAO para cada tabela, como demonstrado nas figuras seguintes.

\begin{figure}[H]
    \centering
    \frame{\includegraphics[width=1\textwidth]{images/Fase3/BaseDeDados/IncendioDAO.png}}
    \caption{Classe IncendioDAO}
\end{figure}

\begin{figure}[H]
    \centering
    \frame{\includegraphics[width=1\textwidth]{images/Fase3/BaseDeDados/LocalizacaoDAO.png}}
    \caption{Classe LocalizacaoDAO}
\end{figure}

\begin{figure}[H]
    \centering
    \frame{\includegraphics[width=1\textwidth]{images/Fase3/BaseDeDados/MeteorologiaDAO.png}}
    \caption{Classe MeteorologiaDAO}
\end{figure}

\begin{figure}[H]
    \centering
    \frame{\includegraphics[width=1\textwidth]{images/Fase3/BaseDeDados/UtilizadorDAO.png}}
    \caption{Classe UtilizadorDAO}
\end{figure}

\section{Povoamento da Base de Dados}

\tab Inicialmente recorremos ao nosso sistema de gestão de base de dados SMSS onde povoamos todas as entidades existentes manualmente, para efeitos de teste.

\tab Posteriormente, decidimos fazer o povoamente da tabela das localizações através de um ficheiro \textit{.json} contendo todas as freguesias de Portugal. Sentimos necessidade de adicionar previamente duas localizações, de forma a poder povoar as tabelas dos incêndios e dos favoritos. Assim que o nosso projeto fosse capaz de recolher os dados dos incêndios de uma \textit{API} e permitisse ao utilizador adicionar localizações favoritas, não precisaríamos deste povoamento prévio.

\tab Quanto ao povoamento das tabelas dos incêndios e da meteorologia, no futuro iríamos recorrer à \textit{API} exposta na segunda fase do projeto.

\tab Em relação aos utilizadores, os mesmos serão registados na base de dados aquando do seu registo na aplicação. 

\begin{figure}[hbt!]
    \centering
    \frame{\includegraphics[width=1\textwidth]{images/Fase3/BaseDeDados/povoamento_Utilizador.png}}
    \caption{Povoamento Utilizador}
\end{figure}

\begin{figure}[hbt!]
    \centering
    \frame{\includegraphics[width=1\textwidth]{images/Fase3/BaseDeDados/povoamento_Meteo_Incendio.png}}
    \caption{Povoamento Meteorologia e Incêndio}
\end{figure}

\begin{figure}[hbt!]
    \centering
    \frame{\includegraphics[width=1\textwidth]{images/Fase3/BaseDeDados/povoamento_Localizacao.png}}
    \caption{Povoamenteo Localização}
\end{figure}

\begin{figure}[hbt!]
    \centering
    \frame{\includegraphics[width=1\textwidth]{images/Fase3/BaseDeDados/povoamento_Favorito.png}}
    \caption{Povoamento Favorito}
\end{figure}

\section{Funcionalidades}

\tab Para demonstrar as funcionalidades da nossa aplicações, vamos apresentar \textit{prints} que expõe as diferentes páginas.

\tab Quando iniciamos a aplicação, somos presenteados com uma página inicial, onde podemos ver um mapa com os incêndios ativos em território nacional. Temos, ainda, a opção de iniciar sessão.

\begin{figure}[H]
    \centering
    \frame{\includegraphics[width=1\textwidth]{images/Fase3/Funcionalidades/pagina_inicial_deslogado.png}}
    \caption{Página inicial - deslogado}
\end{figure}
\clearpage
\tab Caso o utilizador clique em cima de um ícone de incêndio, este é redirecionado para uma página que apresenta as informações mais relevantes do mesmo. A barra de tarefas continua presente, permitindo retroceder à página inicial (clicando no ícone no canto superior esquerdo) ou iniciar sessão.

\begin{figure}[H]
    \centering
    \frame{\includegraphics[width=1\textwidth]{images/Fase3/Funcionalidades/detalhes_incendio.png}}
    \caption{Detalhes de um incêndio ativo}
\end{figure}
\clearpage
\tab Quando o utilizador clica no botão "Iniciar Sessão", este é redirecionado para a página de início de sessão. Nesta, são requisitados o e-mail e a password do utilizador. Em caso de sucesso, é apresentada a página inicial (logado). Caso contrário, a mesma página é recarregada, continuando a pedir os dados de início de sessão.

\tab O utilizador tem, ainda, a opção de se registar na aplicação, clicando na opção "Registar novo utilizador". Neste caso, será redirecionado para a página de registo de um novo utilizador.

\begin{figure}[H]
    \centering
    \frame{\includegraphics[width=1\textwidth]{images/Fase3/Funcionalidades/iniciar_sessao.png}}
    \caption{Página de início de sessão}
\end{figure}
\clearpage
\tab Nesta página, são requisitadas as informações necessárias para criar uma conta na nossa aplicação, autenticando o utilizador. Em caso de sucesso, será retornado para a página de \textit{login}. Caso contrário permanecerá nesta página ou pode voltar à página inicial.

\begin{figure}[H]
    \centering
    \frame{\includegraphics[width=1\textwidth]{images/Fase3/Funcionalidades/registar_utilizador.png}}
    \caption{Página de registo de um utilizador}
\end{figure}
\clearpage
\tab Assim que o utilizador efetua o \textit{login}, a página inicial deixa de ter a opção "Iniciar Sessão" e passa a apresentar o nome do utilizador. Caso este clique em cima do seu nome, é apresentado seu perfil. Em relação ao mapa, este mantém as mesmas funcionalidades da página inicial (deslogado).

\begin{figure}[H]
    \centering
    \frame{\includegraphics[width=1\textwidth]{images/Fase3/Funcionalidades/pagina_inicial_logado.png}}
    \caption{Página inicial - logado}
\end{figure}
\clearpage
\tab A página do perfil do utilizador, apresenta as informações da sua conta. No canto superior direito, o utilizador tem a opção de terminar sessão, voltando à página inicial (deslogado).

\begin{figure}[H]
    \centering
    \frame{\includegraphics[width=1\textwidth]{images/Fase3/Funcionalidades/perfil_utilizador.png}}
    \caption{Perfil do utilizador}
\end{figure}


\clearpage
%============================================================================
% BEGIN CONCLUSÕES E TRABALHO FUTURO
%============================================================================

\chapter{Conclusões e Comentários Finais}
\tab Terminada esta primeira fase do nosso projeto, sendo que será a mais crucial, acreditamos ter conseguido tanto criar uma base sólida para o resto do desenvolvimento do nosso sistema de monitorização como organizar todo o trabalho que ainda nos falta desenvolver.
    
\tab Após a realização da etapa da fundamentação, a equipa conseguiu ter uma melhor perceção daquilo que é necessário fazer para concretizar a tarefa que tem em mãos e, com a observação dos diferentes Diagramas de \textit{Gantt}, a maneira como o trabalho se encontra dividido.
    
\tab Acreditamos, assim, ter conseguido atingir todos os objetivos a que nos propusemos nesta etapa e estar prontos para avançar para a etapa da especificação, onde iremos conceber toda a modelação \textit{UML} do nosso projeto. Esperamos obter, no final de todas as etapas, uma aplicação bem conseguida, acessível a toda a população e que vá de encontro a solucionar vários problemas relacionados ao tema do nosso projeto.

\vspace{0.5cm}
\tab A segunda fase do trabalho foi iniciada com o levantamento e análise de requisitos, que é um passo fundamental para qualquer elaboração de uma \textit{api}. De seguida, e tendo sempre em mente os requisitos obtidos, elaboramos um Modelo de Domínio e os consequentes \textit{Use Cases} da aplicação. Posteriormente, continuamos toda a modelação do projeto em \textit{UML} até à máquina de estados. Finalmente, terminamos esta segunda fase com a conceção da base de dados e um protótipo melhorado da nossa aplicação.

\tab Em suma, pensamos que esta fase do trabalho prático foi bem conseguida e achamos que alcançamos todos os objetivos propostos pelos docentes da unidade curricular. Conseguimos ultrapassar todas as complicações com que nos deparamos, pois sabemos, através de unidades curriculares anteriores, que a modelação em \textit{UML} de um projeto é essencial para o correto funcionamento e idealização do mesmo. 


\vspace{0.5cm}
\tab Na última fase, a equipa colocou as mãos à obra e procedeu então à implementação das funcionalidades e requisitos definidos e idealizados nas etapas anteriores. Apesar de todos se terem esforçado, nem todas as funcionalidades anteriormente idealizadas foram implementadas, como por exemplo o envio de notificações e/ou a consulta do Histórico de Incêndios numa dada Localização.

\tab Após realizada esta fase, a equipa de trabalho passou a entender de forma mais clara a grande importância da premeditação de um horário de trabalho, bem como da organização das várias fases de implementação de um projeto. Posto isto, e apesar de as funcionalidades implementadas serem menos do que as que antecipávamos, achamos que o planeamento feito pela Equipa foi bastante bom. No entanto, entendemos a grande importância de reter sempre algum tempo extra, caso algum processo de implementação não corra da melhor forma. 

\tab Em suma, apesar de pensarmos que bastantes funcionalidades foram implementadas da maneira  mais correta, achamos que podíamos ter realizado ainda mais, tanto mencionadas noutras fases, como acrescentadas nesta, de forma a deixar o projeto no seu melhor estado possível, sempre com uma visão de expansão e de alcançar sempre a nossa melhor versão.

\tab Por fim, achamos importante realçar que o Projeto realizado mostrou à Equipa coisas que esta nunca tinha visto, albergando ensinamentos que a mesma ainda não possuia na àrea da Engenharia e que, sem dúvida alguma, serão fundamentais para a vida estudantil e profissional de cada elemento da Equipa.

\vspace{0.5cm}


%==========================================================================
% END CONCLUSÕES E TRABALHO FUTURO
%==========================================================================

%==========================================================================
% BEGIN REFERÊNCIAS
%==========================================================================

%% Changes biblibography name
%% Portuguese babel default : "Bibliografia"
%% Personally I prefer "Referências"
\renewcommand\bibname{Referências}

%% https://www.overleaf.com/learn/latex/bibliography_management_with_bibtex
\begin{thebibliography}{9}

\bibitem{incendio_pedrogao}
D. de Notícias, “Pedrógão. Chamas mataram 66 pessoas e atingiram cerca de 500 casas,” DN, 17-Jun-2019. [Online]. Available: https://www.dn.pt/pais/pedrogao-grande-chamas-mataram-66-pessoas-e-atingiram-cerca-de-500-casas-11016481.html. [Accessed: 21-Mar-2021]. 

\bibitem{relatorio_fogos_2019} 
MOREIRA, J., PEREIRA, T., CRUZ, M., 2020. Country report for Portugal in San-Miguel-Ayanz et al. (Eds), Forest Fires in Europe, Middle East and North Africa 2019, EUR 30402 EN, Publications Office of the European Union, Luxembourg, 2020, ISBN 978-92-76-23209-4, doi:10.2760/468688, JRC122115

\bibitem{corpos_bombeiros}
“Corpos de Bombeiros,” PORDATA. [Online]. Available: https://www.pordata.pt/Portugal/Corpos+de+Bombeiros-1107. [Accessed: 22-Mar-2021]. 

\end{thebibliography}


%==========================================================================
% END REFERÊNCIAS
%==========================================================================

%==========================================================================
% BEGIN LISTA DE SIGLAS E ACRÓNIMOS
%==========================================================================

%% Portuguese babel does not translate this environment
\renewcommand{\nomname}{Lista de Siglas e Acrónimos}

%% Text that can be shown before acronyms list
\renewcommand{\nompreamble}{}

%% acronyms
\nomenclature[01]{\textbf{API}}{Application Programming Interface}
\nomenclature[02]{\textbf{UML}}{Unified Modeling Language}


%% Show acronyms
\printnomenclature



%==========================================================================
% END LISTA DE SIGLAS E ACRÓNIMOS
%==========================================================================


%==========================================================================
% BEGIN ANEXOS
%==========================================================================

%% Why \addchap, instead of \chapter? 
%% \addchap has no numbering but appears in table of contents.
%\addchap{Anexos}

    %<<Os anexos deverão ser utilizados para a inclusão de informação adicional necessária para uma melhor compreensão do relatório o para complementar tópicos, secções ou assuntos abordados. Os anexos criados deverão ser numerados e possuir uma designação. Estes dados permitirão complementar o Índice geral do relatório relativamente à enumeração e apresentação dos diversos anexos.>>
    
    %% section version of \addchap
    %\addsec{Anexo 1}


%==========================================================================
% END ANEXOS
%==========================================================================
\end{document}